\HeaderA{plot.tile.list}{Plot Dirchlet/Voronoi tiles}{plot.tile.list}
\keyword{hplot}{plot.tile.list}
\begin{Description}\relax
A method for \code{plot}.  Plots (sequentially)
the tiles associated with each point in the set being tessellated.
\end{Description}
\begin{Usage}
\begin{verbatim}
plot.tile.list(x, verbose = FALSE, close=FALSE, pch=1, polycol=NA,
               showpoints=TRUE, asp=1, ...)
## S3 method for class 'tile.list':
plot(x, verbose = FALSE, close=FALSE, pch=1,
                         polycol=NA, showpoints=TRUE, asp=1, ...)
\end{verbatim}
\end{Usage}
\begin{Arguments}
\begin{ldescription}
\item[\code{x}] A list of the tiles in a tessellation, as produced
the function \code{\LinkA{tile.list}{tile.list}()}.
\item[\code{verbose}] Logical scalar; if \code{TRUE} the tiles are
plotted one at a time (with a ``Go?'' prompt after each)
so that the process can be watched.
\item[\code{close}] Logical scalar; if \code{TRUE} the outer edges of
of the tiles (i.e. the edges of the enclosing rectangle)
are drawn.  Otherwise tiles on the periphery of the
tessellation are left ``open''.
\item[\code{pch}] The plotting character for plotting the points of the
pattern which was tessellated.  Ignored if \code{showpoints}
is \code{FALSE}.
\item[\code{polycol}] Optional vector of integers (or \code{NA}s);
the \eqn{i}{}-th entry indicates with which colour to fill
the \eqn{i}{}-th tile.  Note that an \code{NA} indicates
the use of no colour at all.
\item[\code{showpoints}] Logical scalar; if \code{TRUE} the points of
the pattern which was tesselated are plotted.
\item[\code{asp}] The aspect ratio of the plot; integer scalar or
\code{NA}.  Set this argument equal to \code{NA} to allow the data
to determine the aspect ratio and hence to make the plot occupy the
complete plotting region in both \code{x} and \code{y} directions.
This is inadvisable; see the \bold{Warnings}.
\item[\code{...}] Optional arguments; not used.  There for consistency
with the generic \code{plot} function.
\end{ldescription}
\end{Arguments}
\begin{Value}
NULL; side effect is a plot.
\end{Value}
\begin{Section}{Warnings}
The default value for \code{verbose} was formerly \code{TRUE};
it is now \code{FALSE}.

The user is \emph{strongly advised} not to set the value of
\code{asp} but rather to leave \code{asp} equal to its default
value of \code{1}.  Any other value distorts the tesselation
and destroys the perpendicular appearance of lines which are
indeed perpendicular.  (And conversely can cause lines which
are not perpendicular to appear as if they are.)

The argument \code{asp} is present ``just because it can be''.
\end{Section}
\begin{Author}\relax
Rolf Turner
\email{r.turner@auckland.ac.nz}
\url{http://www.math.unb.ca/~rolf}
\end{Author}
\begin{SeeAlso}\relax
\code{\LinkA{tile.list}{tile.list}()}
\end{SeeAlso}
\begin{Examples}
\begin{ExampleCode}
  x <- runif(20)
  y <- runif(20)
  z <- deldir(x,y,rw=c(0,1,0,1))
  w <- tile.list(z)
  plot(w)
  ccc <- heat.colors(20) # Or topo.colors(20), or terrain.colors(20)
                         # or cm.colors(20), or rainbox(20).
  plot(w,polycol=ccc,close=TRUE)
\end{ExampleCode}
\end{Examples}

