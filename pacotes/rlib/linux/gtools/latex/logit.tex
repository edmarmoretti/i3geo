\HeaderA{logit}{Generalized logit and inverse logit function}{logit}
\aliasA{inv.logit}{logit}{inv.logit}
\keyword{math}{logit}
\begin{Description}\relax
Compute generalized logit and generalized inverse logit functions.
\end{Description}
\begin{Usage}
\begin{verbatim}
logit(x, min = 0, max = 1)
inv.logit(x, min = 0, max = 1)
\end{verbatim}
\end{Usage}
\begin{Arguments}
\begin{ldescription}
\item[\code{x}] value(s) to be transformed
\item[\code{min}] Lower end of logit interval
\item[\code{max}] Upper end of logit interval
\end{ldescription}
\end{Arguments}
\begin{Details}\relax
The generalized logit function takes values on [min, max] and
transforms them to span [-Inf,Inf] it is defined as:

\deqn{y = log(\frac{p}{(1-p)})}{y = log(p/(1-p))}

where

\deqn{p=\frac{(x-min)}{(max-min)}}{p=(x-min)/(max-min)}

The generized inverse logit function provides the inverse
transformation:

\deqn{x = p' (max-min) + min}{x = p * (max-min) + min}

where

\deqn{p'=\frac{exp(y)}{(1+exp(y))}}{exp(y)/(1+exp(y))}
\end{Details}
\begin{Value}
Transformed value(s).
\end{Value}
\begin{Author}\relax
Gregory R. Warnes \email{warnes@bst.rochester.edu}
\end{Author}
\begin{SeeAlso}\relax
\code{\LinkA{logit}{logit}}, \code{\LinkA{inv.glogit}{inv.glogit}}
\end{SeeAlso}
\begin{Examples}
\begin{ExampleCode}

  x <- seq(0,10, by=0.25)
  xt <- logit(x, min=0, max=10)
  cbind(x,xt)

  y <- inv.logit(xt, min=0, max=10)
  cbind(x,xt,y)  

\end{ExampleCode}
\end{Examples}

