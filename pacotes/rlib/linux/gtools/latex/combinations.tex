\HeaderA{combinations}{Enumerate the Combinations or Permutations of the Elements of a Vector}{combinations}
\aliasA{permutations}{combinations}{permutations}
\keyword{manip}{combinations}
\begin{Description}\relax
\code{combinations} enumerates the possible combinations of a
specified size from the elements of a vector.  \code{permutations}
enumerates the possible permutations.
\end{Description}
\begin{Usage}
\begin{verbatim}
combinations(n, r, v=1:n, set=TRUE, repeats.allowed=FALSE)
permutations(n, r, v=1:n, set=TRUE, repeats.allowed=FALSE)
\end{verbatim}
\end{Usage}
\begin{Arguments}
\begin{ldescription}
\item[\code{n}] Size of the source vector 
\item[\code{r}] Size of the target vectors 
\item[\code{v}] Source vector. Defaults to \code{1:n}
\item[\code{set}] Logical flag indicating whether duplicates should be
removed from the source vector \code{v}. Defaults to \code{TRUE}.
\item[\code{repeats.allowed}] Logical flag indicating whether the
constructed vectors may include duplicated values.  Defaults to
\code{FALSE}.  
\end{ldescription}
\end{Arguments}
\begin{Details}\relax
Caution: The number of combinations and permutations increases rapidly
with \code{n} and \code{r}!.

To use values of \code{n} above about 45, you will need to increase
R's recursion limit.  See the \code{expression} argument to the
\code{options} command for details on how to do this.
\end{Details}
\begin{Value}
Returns a matrix where each row contains a vector of length \code{r}.
\end{Value}
\begin{Author}\relax
Original versions by Bill Venables
\email{Bill.Venables@cmis.csiro.au}.  Extended to handle
\code{repeats.allowed} by Gregory R. Warnes
\email{warnes@bst.rochester.edu}.
\end{Author}
\begin{References}\relax
Venables, Bill.  "Programmers Note", R-News, Vol 1/1,
Jan. 2001. \url{http://cran.r-project.org/doc/Rnews}
\end{References}
\begin{SeeAlso}\relax
\code{\LinkA{choose}{choose}}, \code{\LinkA{options}{options}}
\end{SeeAlso}
\begin{Examples}
\begin{ExampleCode}
combinations(3,2,letters[1:3])
combinations(3,2,letters[1:3],repeats=TRUE)

permutations(3,2,letters[1:3])
permutations(3,2,letters[1:3],repeats=TRUE)

# To use large 'n', you need to change the default recusion limit
options(expressions=1e5)
cmat <- combinations(300,2)
dim(cmat) # 44850 by 2 
\end{ExampleCode}
\end{Examples}

